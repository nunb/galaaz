\documentclass[]{article}
\usepackage{lmodern}
\usepackage{amssymb,amsmath}
\usepackage{ifxetex,ifluatex}
\usepackage{fixltx2e} % provides \textsubscript
\ifnum 0\ifxetex 1\fi\ifluatex 1\fi=0 % if pdftex
  \usepackage[T1]{fontenc}
  \usepackage[utf8]{inputenc}
\else % if luatex or xelatex
  \ifxetex
    \usepackage{mathspec}
  \else
    \usepackage{fontspec}
  \fi
  \defaultfontfeatures{Ligatures=TeX,Scale=MatchLowercase}
\fi
% use upquote if available, for straight quotes in verbatim environments
\IfFileExists{upquote.sty}{\usepackage{upquote}}{}
% use microtype if available
\IfFileExists{microtype.sty}{%
\usepackage{microtype}
\UseMicrotypeSet[protrusion]{basicmath} % disable protrusion for tt fonts
}{}
\usepackage[margin=1in]{geometry}
\usepackage{hyperref}
\hypersetup{unicode=true,
            pdftitle={High Quality Scientific Plotting with Ruby in GraalVM},
            pdfauthor={Rodrigo Botafogo},
            pdfborder={0 0 0},
            breaklinks=true}
\urlstyle{same}  % don't use monospace font for urls
\usepackage{color}
\usepackage{fancyvrb}
\newcommand{\VerbBar}{|}
\newcommand{\VERB}{\Verb[commandchars=\\\{\}]}
\DefineVerbatimEnvironment{Highlighting}{Verbatim}{commandchars=\\\{\}}
% Add ',fontsize=\small' for more characters per line
\usepackage{framed}
\definecolor{shadecolor}{RGB}{248,248,248}
\newenvironment{Shaded}{\begin{snugshade}}{\end{snugshade}}
\newcommand{\KeywordTok}[1]{\textcolor[rgb]{0.13,0.29,0.53}{\textbf{#1}}}
\newcommand{\DataTypeTok}[1]{\textcolor[rgb]{0.13,0.29,0.53}{#1}}
\newcommand{\DecValTok}[1]{\textcolor[rgb]{0.00,0.00,0.81}{#1}}
\newcommand{\BaseNTok}[1]{\textcolor[rgb]{0.00,0.00,0.81}{#1}}
\newcommand{\FloatTok}[1]{\textcolor[rgb]{0.00,0.00,0.81}{#1}}
\newcommand{\ConstantTok}[1]{\textcolor[rgb]{0.00,0.00,0.00}{#1}}
\newcommand{\CharTok}[1]{\textcolor[rgb]{0.31,0.60,0.02}{#1}}
\newcommand{\SpecialCharTok}[1]{\textcolor[rgb]{0.00,0.00,0.00}{#1}}
\newcommand{\StringTok}[1]{\textcolor[rgb]{0.31,0.60,0.02}{#1}}
\newcommand{\VerbatimStringTok}[1]{\textcolor[rgb]{0.31,0.60,0.02}{#1}}
\newcommand{\SpecialStringTok}[1]{\textcolor[rgb]{0.31,0.60,0.02}{#1}}
\newcommand{\ImportTok}[1]{#1}
\newcommand{\CommentTok}[1]{\textcolor[rgb]{0.56,0.35,0.01}{\textit{#1}}}
\newcommand{\DocumentationTok}[1]{\textcolor[rgb]{0.56,0.35,0.01}{\textbf{\textit{#1}}}}
\newcommand{\AnnotationTok}[1]{\textcolor[rgb]{0.56,0.35,0.01}{\textbf{\textit{#1}}}}
\newcommand{\CommentVarTok}[1]{\textcolor[rgb]{0.56,0.35,0.01}{\textbf{\textit{#1}}}}
\newcommand{\OtherTok}[1]{\textcolor[rgb]{0.56,0.35,0.01}{#1}}
\newcommand{\FunctionTok}[1]{\textcolor[rgb]{0.00,0.00,0.00}{#1}}
\newcommand{\VariableTok}[1]{\textcolor[rgb]{0.00,0.00,0.00}{#1}}
\newcommand{\ControlFlowTok}[1]{\textcolor[rgb]{0.13,0.29,0.53}{\textbf{#1}}}
\newcommand{\OperatorTok}[1]{\textcolor[rgb]{0.81,0.36,0.00}{\textbf{#1}}}
\newcommand{\BuiltInTok}[1]{#1}
\newcommand{\ExtensionTok}[1]{#1}
\newcommand{\PreprocessorTok}[1]{\textcolor[rgb]{0.56,0.35,0.01}{\textit{#1}}}
\newcommand{\AttributeTok}[1]{\textcolor[rgb]{0.77,0.63,0.00}{#1}}
\newcommand{\RegionMarkerTok}[1]{#1}
\newcommand{\InformationTok}[1]{\textcolor[rgb]{0.56,0.35,0.01}{\textbf{\textit{#1}}}}
\newcommand{\WarningTok}[1]{\textcolor[rgb]{0.56,0.35,0.01}{\textbf{\textit{#1}}}}
\newcommand{\AlertTok}[1]{\textcolor[rgb]{0.94,0.16,0.16}{#1}}
\newcommand{\ErrorTok}[1]{\textcolor[rgb]{0.64,0.00,0.00}{\textbf{#1}}}
\newcommand{\NormalTok}[1]{#1}
\usepackage{graphicx,grffile}
\makeatletter
\def\maxwidth{\ifdim\Gin@nat@width>\linewidth\linewidth\else\Gin@nat@width\fi}
\def\maxheight{\ifdim\Gin@nat@height>\textheight\textheight\else\Gin@nat@height\fi}
\makeatother
% Scale images if necessary, so that they will not overflow the page
% margins by default, and it is still possible to overwrite the defaults
% using explicit options in \includegraphics[width, height, ...]{}
\setkeys{Gin}{width=\maxwidth,height=\maxheight,keepaspectratio}
\IfFileExists{parskip.sty}{%
\usepackage{parskip}
}{% else
\setlength{\parindent}{0pt}
\setlength{\parskip}{6pt plus 2pt minus 1pt}
}
\setlength{\emergencystretch}{3em}  % prevent overfull lines
\providecommand{\tightlist}{%
  \setlength{\itemsep}{0pt}\setlength{\parskip}{0pt}}
\setcounter{secnumdepth}{5}
% Redefines (sub)paragraphs to behave more like sections
\ifx\paragraph\undefined\else
\let\oldparagraph\paragraph
\renewcommand{\paragraph}[1]{\oldparagraph{#1}\mbox{}}
\fi
\ifx\subparagraph\undefined\else
\let\oldsubparagraph\subparagraph
\renewcommand{\subparagraph}[1]{\oldsubparagraph{#1}\mbox{}}
\fi

%%% Use protect on footnotes to avoid problems with footnotes in titles
\let\rmarkdownfootnote\footnote%
\def\footnote{\protect\rmarkdownfootnote}

%%% Change title format to be more compact
\usepackage{titling}

% Create subtitle command for use in maketitle
\newcommand{\subtitle}[1]{
  \posttitle{
    \begin{center}\large#1\end{center}
    }
}

\setlength{\droptitle}{-2em}

  \title{High Quality Scientific Plotting with Ruby in GraalVM}
    \pretitle{\vspace{\droptitle}\centering\huge}
  \posttitle{\par}
  \subtitle{Also: Allowing R to use classes, modules, blocks, etc.}
  \author{Rodrigo Botafogo}
    \preauthor{\centering\large\emph}
  \postauthor{\par}
      \predate{\centering\large\emph}
  \postdate{\par}
    \date{19 October 2018}

% usar portugues do Brasil
% \usepackage[brazilian]{babel}
\usepackage[utf8]{inputenc}

\usepackage{geometry}
\geometry{a4paper, top=1in}

% necessários para uso com kableExtra
\usepackage{longtable}
\usepackage{multirow}
\usepackage[table]{xcolor}
\usepackage{wrapfig}
\usepackage{float}
\usepackage{colortbl}
\usepackage{pdflscape}
\usepackage{tabu}
\usepackage{threeparttable}
\usepackage[normalem]{ulem}

\usepackage{bbm}
\usepackage{booktabs}
\usepackage{expex}

\usepackage{graphicx}
\usepackage{fancyhdr}
\pagestyle{fancy}
\fancyhf{}

\usepackage{lipsum}

% disponibilizar o comando lastpage
\usepackage{lastpage}

% tamanho do font padrão 11pt
\usepackage[fontsize=10pt]{scrextend}

% comandos para formatar uma tabela
\usepackage{array}
\newcolumntype{L}[1]{>{\raggedright\let\newline\\\arraybackslash\hspace{0pt}}m{#1}}
\newcolumntype{C}[1]{>{\centering\let\newline\\\arraybackslash\hspace{0pt}}m{#1}}
\newcolumntype{R}[1]{>{\raggedleft\let\newline\\\arraybackslash\hspace{0pt}}m{#1}}

% necessário para importar outros arquivos latex
\usepackage{import}

\newcommand{\RtoLatex}[2]{\newcommand{#1}{#2}}
%\newcommand{\atraso}[1]{\color{red} \textbf {Tempo desde a Assinatura do Contrato: #1 dias}}

\begin{document}
\maketitle

\section{Introduction}\label{introduction}

Ruby is a dynamic, interpreted, reflective, object-oriented,
general-purpose programming language. It was designed and developed in
the mid-1990s by Yukihiro ``Matz'' Matsumoto in Japan. It reached high
popularity with the development of Ruby on Rails (RoR) by David
Heinemeier Hansson. RoR is a web application framework which was first
release circa 2005 and makes extensive use of Ruby's metaprogramming
features. With the advend of RoR, Ruby became extremely popular and it
peeked in popularity around 2008 according to the Tiobe index
(\url{https://www.tiobe.com/tiobe-index/ruby/}). From 2008 to 2015, it's
popularity declined consistently and then started picking up again
during the next 3 years. At the time of this writing (November 2018),
Ruby is ranked 16th in the Tiobe index.

Python, considered a similar language to Ruby with similar features
ranks 4th in the index. The first three positions are taken by Java, C
and C++. One criticism often heard about Ruby, is that it is useful only
for web applications while Python, with similar features has more
diverse libraries, being useful for web applications with the Django
framework, but also for scientific applications such as statistics, data
analysis, big data, biology, etc. This criticism is by no way wrong.
Although Ruby can do much more than just web applications:
\url{https://github.com/markets/awesome-ruby}, for scientific computing,
Ruby lags way behind Python and R, the two most prestigous languages in
the field, with R being prefered by statisticians while Python is
prefered by everyone else, because of it's gentle learning curve and
more ``natural'' programming paradigm.

Until recently, there was no real perspective for Ruby to bridge this
gap and have even the most basic scientific computing infrastructure.
Comes GraalVM into the picture:

\begin{verbatim}
 GraalVM is a universal virtual machine for running applications written in JavaScript,
 Python 3, Ruby, R, JVM-based languages like Java, Scala, Kotlin, and LLVM-based languages
 such as C and C++.

 GraalVM removes the isolation between programming languages and enables interoperability in a
 shared runtime. It can run either standalone or in the context of OpenJDK, Node.js,
 Oracle Database, or MySQL.

 GraalVM allows you to write polyglot applications with a seamless way to pass values from one
 language to another. With GraalVM there is no copying or marshaling necessary as it is with
 other polyglot systems. This lets you achieve high performance when language boundaries are
 crossed. Most of the time there is no additional cost for crossing a language boundary at all.

 Often developers have to make uncomfortable compromises that require them to rewrite
 their software in other languages. For example:

  * “That library is not available in my language. I need to rewrite it.” 
  * “That language would be the perfect fit for my problem, but we cannot run it
    in our environment.” 
  * “That problem is already solved in my language, but the language is too slow.”

With GraalVM we aim to allow developers to freely choose the right language for the task at
hand without making compromises.
\end{verbatim}

As stated above, GraalVM is a \emph{universal} virtual machine that
allows Ruby and R (and other languages) to run on the same environment.
GraalVM allows polyglot applications to \emph{seamlessly} interact with
one another and pass values from one language to the other. Based on
GraalVM, the Galaaz project was started. Galaaz indends to tightly
couple Ruby and R and allow those languages to \emph{seamlessly}
interact in a way that the user will be unaware of such interaction.

Library wrapping is an usual way of bringing features from one library
into another language. For instance, whenever Python needs to perform
operations efficiently, C libraries are wrapped in Python. For the
Python developer, the existence of such C library is of no concern. The
problem with library wrapping is that for any new library of interest,
there is the need to hand craft a new wrapper. With Galaaz, the same
concept of wrapping was done, but instead of wrapping a single C or R
library, Galaaz wraps the whole of the R language. Doing so, all
thousands of R libraries are immediately available to Ruby developers
and any new library developed in R will also become available without
requiring a new wrapping effort.

In this article, the graphing ggplot2 library from R will be accessed by
Ruby transparently, bringing to Ruby the power of high quality
scientific plotting. It might seem, from the exposed above, that Galaaz
mainly benefits Ruby developers and might be of no consequence to the R
developer. This article will however show that migrating from R to Ruby
with Galaaz is a matter of small syntactic changes. Furthermore, R lacks
some fundamental constructs for code reuse and large system
construction. Using Galaaz, the R developer can easily migrate to a
powerful OO language, at virtually no cost and then, as needs requires,
she can add them to her toolbox.

In this article we will explore the R ToothGrowth dataset. In doing so,
we will create some plots. Furthermore we will create a ``Corporate
Template'' for our plots ensuring that any plot of the same type will
have a consistent visualisation.

\section{gKnit}\label{gknit}

This document was written using rmarkdown and the corresponding HTML was
generated by the gKnit application. gKnit is a wrapper around the
powerful `knitr' application which converts rmarkdown text to many
different output formats such as HTML, Latex, docx, etc. The gKnit tool
is still under active development and will soon be released.

In rmarkdown, text and code can be part of the same document, and code
blocks are marked with a special markup. Interested readers can easily
google `knitr' and `rmarkdown'. in gKnit, each Ruby block is evaluated
independently and `eval' in Ruby creates a new scope, so, in order for a
variable defined in a block to be accessible in another block, it has to
be a global variable, preceded by the `\$' sign.

\section{Exploring the Dataset}\label{exploring-the-dataset}

Let start by exploring our selected dataset. In this dataset the
response is the length of odontoblasts (cells responsible for tooth
growth) in 60 guinea pigs. Each animal received one of three dose levels
of vitamin C (0.5, 1, and 2 mg/day) by one of two delivery methods,
orange juice or ascorbic acid (a form of vitamin C and coded as VC).

In Galaaz, in order to have access to an R variable pointed by an R
symbol we use the corresponding Ruby symbol preceeded by the tilda
(`\textasciitilde{}') function.

\begin{Shaded}
\begin{Highlighting}[]
\CommentTok{# Read the R ToothGrowth variable and assign it to the}
\CommentTok{# Ruby tooth_growth variable}
\DataTypeTok{$tooth_growth}\NormalTok{ = ~}\StringTok{:ToothGrowth}
\CommentTok{# convert the dose to a factor}
\DataTypeTok{$tooth_growth}\NormalTok{.dose = }\DataTypeTok{$tooth_growth}\NormalTok{.dose.as__factor}

\CommentTok{# print the first few elements of the dataset}
\NormalTok{puts }\DataTypeTok{$tooth_growth}\NormalTok{.head}
\end{Highlighting}
\end{Shaded}

\begin{verbatim}
##    len supp dose
## 1  4.2   VC  0.5
## 2 11.5   VC  0.5
## 3  7.3   VC  0.5
## 4  5.8   VC  0.5
## 5  6.4   VC  0.5
## 6 10.0   VC  0.5
\end{verbatim}

Great! We've managed to read the ToothGrowth dataset and take a look at
its elements. Observe that we have three columns in this dataset: `len',
`supp' and `dose'. Accessing a column, for example the `len' column, is
done by doing `\$tooth\_growth.len'.

Let's explore some more details of this dataset. In particular, let's
look at its dimensions, structure and summary statistics.

\begin{Shaded}
\begin{Highlighting}[]
\NormalTok{puts }\DataTypeTok{$tooth_growth}\NormalTok{.dim}
\CommentTok{# chdck why NULL}
\NormalTok{puts R.str(}\StringTok{:ToothGrowth}\NormalTok{)}
\NormalTok{puts }\DataTypeTok{$tooth_growth}\NormalTok{.summary}
\end{Highlighting}
\end{Shaded}

\begin{verbatim}
## [1] 60  3
## NULL
##       len        supp     dose   
##  Min.   : 4.20   OJ:30   0.5:20  
##  1st Qu.:13.07   VC:30   1  :20  
##  Median :19.25           2  :20  
##  Mean   :18.81                   
##  3rd Qu.:25.27                   
##  Max.   :33.90
\end{verbatim}

Let's now create our first plot with the given data by accessing ggplot2
from Ruby. For Rubyist that have never seen or used ggplot2, here is the
description found on ggplot home page:

\begin{verbatim}
"ggplot2 is a system for declaratively creating graphics, based on _The Grammar of Graphics_.
You provide the data, tell ggplot2 how to map variables to aesthetics, what graphical 
primitives to use, and it takes care of the details."
\end{verbatim}

This description might be a bit cryptic and it is best to see it at work
to understand it. Basically, in the \emph{grammar of graphics} each
component of the plot such as the grid, the axis, the data, title,
subtitle, etc. is added to the plot in layers to form the final
graphics.

In this plot bellow, the `dose' is plotted on the `x' axis and the tooth
length on the `y' axis. Note the specification in the the `aes' method:
`E.aes(x: :dose, y: :len)', where `:dose' is the `dose' column of the
dataset and `:len' the `len' column. The `aes' method is the
\emph{aesthetics} for this plot. Then, to this layer, the
`geom\_boxplot' is added and the whole plot is printed.

Note also that we have a call to `R.png' before plotting and
'R.dev\_\_off' after the print statement. `R.png' opens a `png' device
for writing the plot. When 'R.dev\_\_off' is called, the device is
closed and a `png' file is created. If no name is given to the `png'
function, a file named `Rplot' is generated, where is the number of the
plot. So, this first plot is called `Rplot001.png'. We can then include
the generated `png' file in this document, by adding an rmarkdown
directive.

\begin{Shaded}
\begin{Highlighting}[]
\NormalTok{require }\StringTok{'ggplot'}

\NormalTok{R.png}

\NormalTok{e = }\DataTypeTok{$tooth_growth}\NormalTok{.ggplot(E.aes(}\StringTok{x: :dose}\NormalTok{, }\StringTok{y: :len}\NormalTok{))}
\NormalTok{print e + R.geom_boxplot}

\NormalTok{R.dev__off}
\end{Highlighting}
\end{Shaded}

\begin{figure}
\centering
\includegraphics{Rplot001.png}
\caption{ToothGrowth}
\end{figure}

We've just managed to generate our first plot in Ruby with only two
lines of code. This plot, however, if far from being pleasing to the
eye.

\section{Conclusion}\label{conclusion}

\section{Installing Galaaz}\label{installing-galaaz}

\subsection{Prerequisites}\label{prerequisites}

\begin{itemize}
\tightlist
\item
  GraalVM (\textgreater{}= rc8)
\item
  TruffleRuby
\item
  FastR
\end{itemize}

The following R packages will be automatically installed when necessary,
but could be installed prior to using gKnit if desired:

\begin{itemize}
\tightlist
\item
  ggplot2
\item
  gridExtra
\item
  knitr
\end{itemize}

Installation of R packages requires a development environment and can be
time consuming. In Linux, the gnu compiler and tools should be enough. I
am not sure what is needed on the Mac.

\subsection{Preparation}\label{preparation}

\begin{itemize}
\tightlist
\item
  gem install galaaz
\end{itemize}

\subsection{Usage}\label{usage}

\begin{itemize}
\tightlist
\item
  gknit 
\end{itemize}


\end{document}
